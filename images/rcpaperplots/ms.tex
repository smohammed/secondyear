\documentclass[twocolumn]{emulateapj}
\usepackage{amsmath}
\usepackage{lipsum}
\usepackage{graphicx}
\shorttitle{Red Clump Stars in GALEX and Gaia}
\shortauthors{MOHAMMED ET AL.}
\usepackage{hyperref}



\def\deg{\ifmmode^\circ\else$^\circ$\fi} 

\begin{document}

\title{An Ultraviolet-Optical Color-Metallicity relation for Red Clump Stars using GALEX and Gaia}

\author{Steven Mohammed}
\affiliation{Department of Astronomy, Columbia University, New York, NY 10027}

\author{David Schiminovich}
\affiliation{Department of Astronomy, Columbia University, New York, NY 10027} 

\author{Keith Hawkins}
\affiliation{Department of Astronomy, Columbia University, New York, NY 10027}
\affiliation{Department of Astronomy, The University of Texas at Austin, 2515 Speedway Boulevard Stop C1400, Austin, TX 78712}

\author{Benjamin Johnson}
\affiliation{Harvard-Smithsonian Center for Astrophysics, Cambridge, MA 02138}

\author{Dun Wang}
\affiliation{Center for Cosmology and Particle Physics, Department of Physics, New York University, New York, NY 10003}, 

\author{David W. Hogg}
\affiliation{Center for Cosmology and Particle Physics, Department of Physics, New York University, New York, NY 10003}



\email{smohammed@astro.columbia.edu}

\begin{abstract}
Although core helium-burning red clump (RC) stars are faint at ultraviolet wavelengths, their ultraviolet-optical color is a unique and accessible probe of their physical properties. Using data from the GALEX All Sky Imaging Survey, Gaia Data Release 2 and the SDSS APOGEE DR14 survey, we find that spectroscopic metallicity is strongly correlated with the location of an RC star in the UV-optical color magnitude diagram.  The RC has a wide spread in (NUV - G)$_0$ color, over 4 magnitudes, compared to a 0.7-magnitude range in (G$_{BP}$ - G$_{RP}$)$_0$. We propose a photometric, dust-corrected, ultraviolet-optical (NUV - G)$_0$ color-metallicity [Fe/H] relation using a sample of 5,175 RC stars from APOGEE. We show that this relation has a scatter of 0.28 dex and is easier to obtain for large, wide-field samples than spectroscopic metallicities. Importantly, the effect may be comparable to the spread in RC color attributed to extinction in other studies.
\end{abstract}
 
\keywords{catalogs, ultraviolet: stars, stars: evolution, Galaxy: general}

\section{Introduction}
The Milky Way is host to a variety of stars spanning the entire stellar lifetime range. Average stars like our Sun eventually become red giants, some of which populate a prominent feature in color-magnitude diagrams (CMDs) called the red clump (RC) consisting of low-mass, metal-rich stars in the core helium-burning stage of stellar evolution. A sizable fraction of Solar neighborhood giants observed with Hipparcos are RC candidates (60$\%$, \citealt{girardi16}). Metallicities for these stars are readily available from surveys such as APOGEE. The metallicity of RC stars can be used to understand the star formation history and ages of stars in the Milky Way and inform stellar evolutionary models of RC, red giant and horizontal branch stars (\citealt{girardi16}).

\begin{figure*}[] % Fig1 -  MG vs NUVG CMD
\centering
\includegraphics[width=0.8\textwidth]{f1.pdf}
\caption{M$_G$ vs (NUV - G)$_0$ distribution for the full parallax-selected ($<3.5$ kpc) GAIS-Gaia catalog  (top panels) with matches with RC stars shown (bottom panels). The left panels show the CMD without a Galactic extinction correction and the right panels apply a correction as described in the text. The main locus at the center of each panel shows the main sequence.  The RC stars are overlaid and colored by [Fe/H]. There is a clear trend of metallicity with NUV - G color in both cases. [Fe/H] from APOGEE DR14.}
\end{figure*}

\begin{figure*}[] % Fig2 -MG vs BPRP CMD
\centering
\includegraphics[width=0.8\textwidth]{f2.pdf}
\caption{M$_G$ vs G$_{BP}$ - G$_{RP}$ distribution for the full GAIS-Gaia catalog (top panels) with matches with RC stars shown (bottom panels). The left panels show the CMD without a Galactic extinction correction and the right panels apply a correction as described in the text.}
\end{figure*}

% 5. The introduction might benefit from a few sentences on how the properties of the stars (ie colors and temperature) in the core helium burning phase depend on mass, metallicity, alpha enhancement, age, etc.

\textbf{Effective temperature is a proxy for color. T$_{\rm{eff}}$ and the luminosity of a RC star remain relatively constant throughout most of its life, getting smaller as the star gets older for a constant [Fe/H]. T$_{\rm{eff}}$ is dependant on the initial mass of the RC star (\citealt{girardi16}, Figure 1a) and also will greatly affect the metallicity; the higher the temperature, the higher Z is likely to be. Depending on the distance of the star, the relation between [$\alpha$/Fe] and [Fe/H] will differ, even showing some bimodality in regions above the disk (\citealt{hayden15}). For a given total stellar mass, the metallicity Z of a RC star varies with luminosity. Stars with a lower Z will have a higher luminosity and vice versa.}

% 4. You mention in Section 4 that this is not the first attempt to use the photometry of stars to infer their metallicity. Please add some discussion of these previous attempts into the introduction, especially those that focused on the red clump. It might also be valuable to directly compare these results to what you are doing here, to show how what you are doing is better.

\textbf{Several attempts to relate photometric measurements to metallicities have been made. \citealt{ruiz18} use photometry from Gaia DR1, Hipparcos, Tycho 2, 2MASS, APASS DR9, and WISE and metallicities from various sources to calibrate a dereddened HR diagram. \citealt{ivezic2008} attempt to create a metallicity map of the Milky Way using only SDSS photometry that have similar errors to properties measured from SDSS spectra. \citealt{cole2000} use Str\"{o}mgren vby photometry to obtain a color-metallicity relation for red giants in the Large Magellanic Cloud and find that it is comparable with their spectroscopic results but practically useful for quickly obtaining metallicities for many objects.}

With the release of Gaia DR2 (\citealt{gaia}), the Milky Way can now be probed to greater depths than ever before. The Gaia DR2 release presents parallax measurements for over 1 billion stars, which provide crucial distance information and G-band measurements to allow construction of the CMD of the Milky Way field population. We combine NUV-band data from the GALEX All Sky Imaging Survey (GAIS, \citealt{galex}) with Gaia and a catalog of RC stars from APOGEE DR14 (\citealt{ting18}). RC stars are very faint in NUV and should separate clearly from the main sequence. As RC stars have not been extensively probed in UV, this study has the potential to strengthen our understanding the relation between UV-optical colors and the physical properties of stars in this core helium-burning stage. In this paper, we focus on a (NUV - G)$_0$ color-metallicity relation and show how it compares to a similar color-metallicity relation derived using optical colors. Finally, we also discuss how this relation compares to predictions from stellar evolutionary model tracks, using the MIST code (\citealt{mist}, \citealt{choi16}, \citealt{paxton11}, \citealt{paxton13}, \citealt{paxton15}).

\begin{figure}[] % Fig3 - RC histograms
\centering
\includegraphics[width=0.5\textwidth]{f3.pdf}
\caption{Various statistics for our GAIS-Gaia RC sample. The sample is mostly restricted to the upper Galactic plane and at distances greater than 500 pc. We also overplot the low and high [$\alpha$/Fe] populations discussed further below.}
\end{figure}

\section{Observations}
\subsection{Red Clump data}
We build a sample of 5,175 RC stars from \citealt{ting18} which is constructed using data from the APOGEE (Apache Point Observatory Galactic Evolution Experiment, \citealt{apogee2017}) and LAMOST (Large Sky Area Multi-Object Fibre Spectroscopic Telescope, \citealt{lamost}) surveys. \citealt{ting18} build a RC sample of Milky Way stars from APOGEE DR14 data with $\sim$ 3$\%$ contamination from red giant stars. For our analysis we only used the pristine RC sample obtained from APOGEE spectra. From its high signal-to-noise, near-infrared (1.51-1.70 $\mu$m) spectra, derived parameters such as metallicities, T$_{\rm{eff}}$, log g parameters are available, as well as abundances for many elements. APOGEE elemental abundances are typically accurate to 0.2 dex over the metallicity range considered here (\citealt{ASPCAP}).


\begin{figure}[] % Fig4 - Teff vs color
\centering
\includegraphics[width=0.5\textwidth]{f4.pdf}
\caption{T$_{\rm{eff}}$ from APOGEE vs extinction corrected (G$_{BP}$ - G$_{RP}$)$_0$ and (NUV - G)$_0$ colored by [Fe/H] for our RC sample. (G$_{BP}$ - G$_{RP}$)$_0$ shows significant scatter.  (NUV - G)$_0$ is more tightly correlated with T$_{\rm{eff}}$, although we also identify a subpopulation of very blue outliers.}
\end{figure}

\subsection{GAIS and Gaia DR2}
The GALEX All Sky Imaging Survey (GAIS) contains NUV data for millions of objects across the entire sky. Gaia DR2 provides Gaia G, G$_{BP}$, and G$_{RP}$ magnitudes and parallaxes that can be used to obtain distance information. The errors in G, G$_{BP}$, and G$_{RP}$ are of the order of millimagnitudes. We apply a small error-dependent correction to the Gaia parallaxes (\citealt{lutzkelker73}, \citealt{Oudmaijer98}). We then invert the parallax to get a distance. To cross-match these data we use the astropy (\citealt{astropy}) function \textit{search around sky} with a search radius of 3 arcseconds. In total we utilize coverage in GALEX NUV, Gaia G, G$_{BP}$, G$_{RP}$ and the relevant APOGEE footprint.

Galactic extinction plays a much larger role for the NUV than the other bands (\citealt{ccm89}) and will have a nontrivial effect on the location of objects in a CMD. To account for this reddening we use the 3D dust map from \citealt{GSF15}, which gives E$_{B - V}$ as a function of distance, in conjunction with Gaia parallax-derived distances to estimate the reddening in the line of sight of each object in this catalog. The NUV - G color is dust-corrected (indicated by a 0 subscript) using these E$_{B - V}$ values, adopting R$_{NUV}$ from \citealt{yuan13}, and R$_G$ from \citealt{jordi2010} who obtain R$_G$ values between 2.4 and 3.6. For our analysis, the extinction corrections for NUV and G are NUV$_0$ = NUV - E$_{B - V}$ $\times$ 7.24 and G$_0$ = G - E$_{B - V}$ $\times$ 2.85. We are restricted to the sky coverage of the \citealt{GSF15} map and remove any objects in the GAIS-Gaia catalog that do not overlap with the map. 

For the final catalog we make several additional cuts to the data. The final catalog contains objects that have detections in NUV, G, G$_{BP}$, and G$_{RP}$, [Fe/H] and T$_{\rm{eff}}$ measurements, parallax errors less than 10$\%$, $visibility$\textunderscore$periods$\textunderscore$used$ $>$ 8, and distances less than 3500 pc. Additionally, we use the $RC$\textunderscore$Pristine$ Classification from \citealt{ting18}. We do not require the APOGEE flags to be set to 0 for these objects (65 in total) however the removal of these objects do not impact our results. Our final catalog of GAIS and Gaia objects contains 10,357,542 objects. We cross-match this catalog with the RC catalog and obtain 5,175 matches. 

GAIS-Gaia does not appear to be limited to only very blue objects despite the expectation that GALEX would not observe many red stars. There is a large population at the expected position of the RC in the CMD. There are about 91$\%$ of objects in the main catalog along the Main Sequence versus 4$\%$ of objects in the RC. 

\begin{figure}[] % Fig5 - Alphafe vs FeH
\centering
\includegraphics[width=0.5\textwidth]{f5.pdf}
\caption{[$\alpha$/Fe] vs [Fe/H] colored by the dust corrected (NUV - G)$_0$ (described in Figure 1). \citealt{hawkins15} discuss a cartoon depiction of this trend separating the lower branch (low [$\alpha$/Fe] stars) from the upper branch (High [$\alpha$/Fe] stars). \textbf{The lines drawn here are just a qualitative cut based on the emergence of two different populations combined with work from \citealt{hawkins15}.}}
\end{figure}

\section{Results}   
Figures 1 and 2 show UV-optical and optical CMD histograms for the GAIS-Gaia catalog, using both uncorrected and Galactic extinction-corrected magnitudes. The general shape of the UV-optical CMD is very similar to that of optical: a large main sequence with the red giant branch and RC prominently displayed. The main sequence stretches from (NUV - G)$_0$ = 8 and M$_G$ = 6 to (NUV - G)$_0$ = 2.5 and M$_G$ = 1, and is where the bulk of the survey matches appear. The secondary locus around (NUV - G)$_0$ = 8 and M$_G$ = 0 (\citealt{hawkins17}) is populated by red giants, notably RC stars. The spread of the entire RC in (NUV - G)$_0$ is unlike that seen in (G$_{BP}$ - G$_{RP}$)$_0$, spreading over 4 magnitudes compared to a spread of 0.7 magnitudes in (G$_{BP}$ - G$_{RP}$)$_0$, as shown in Figure 2. The spread could be due to the age, metallicity, and extinction of the RC stars. As discussed further below, our optical CMD is similar to that of \citealt{gaiahrd} that shows the appearance of a RC for low extinction sources (E(B - V) $<$ 0.015) in DR2.

\textbf{We show a portion of our catalog-matched RC sample in Table 1.} In Figures 1 and 2 we overplot this sample in the bottom panels. The RC stars show a clear NUV-optical color-dependent trend with [Fe/H] (higher [Fe/H] at redder (NUV - G)$_0$ and vice-versa, shown in Figure 1). This spread is unique to (NUV - G)$_0$ color, especially when the dust correction is applied. In Figure 3 we show the distributions for the RC stars of Galactic longitude, distance, NUV magnitude and E(B - V). The E(B - V) for most of this sample is less than 0.1, suggesting relatively small extinction corrections. Most of the RC sample is between 18 $<$ NUV $<$ 20. The GAIS survey limit is 21 (5$\sigma$) indicating this sample is reasonably complete out to our distance limit. 

Using the derived parameters from APOGEE, we show in Figure 4 that the RC UV-optical color also correlates with effective temperature. The trend for (G$_{BP}$ - G$_{RP}$)$_0$ is much weaker and more highly scattered. From a line fit, we measure $\sigma_{{(BP-RP)}_0}$ = 0.11 and $\sigma_{{(NUV-G)}_0}$ = 0.43. \textbf{Compared to the slope of the RC in each color, the relative scatter is 3.3 times smaller in (NUV - G)$_0$ than in (G$_{BP}$ - G$_{RP}$)$_0$. If we simply look at the coefficient of determination, $r^2_{(BP-RP)_0}$ = 0.15 while $r^2_{(NUV-G)_0}$ = 0.66.}


\section{Discussion}
\subsection{Spectroscopic Catalog-matched Sample}
In this section we define an ultraviolet-optical color-metallicity relation. First we separate our RC sample into two subsamples of low and high [$\alpha$/Fe] stars. Figure 5 shows [$\alpha$/Fe] vs [Fe/H] and distinguishes between low and high [$\alpha$/Fe] stars using a simple cut in [$\alpha$/Fe] \textbf{and [Fe/H] (\citealt{hawkins15}, \citealt{nidever14}, \citealt{li18})}. The high [$\alpha$/Fe] stars overall have a much higher [$\alpha$/Fe], especially at lower [Fe/H]. The [$\alpha$/Fe] - [Fe/H] relation is a unique way to separate different Galactic components to understand the star formation history of different parts of the Milky Way. The low [$\alpha$/Fe] stars are thought to be considerably younger than the high [$\alpha$/Fe] stars because of the smaller amount of $\alpha$ elements (O, Mg, Si, Ca, and Ti) for a given [Fe/H]. We see the same separation as in \citealt{nidever14} between the low and high $\alpha$ sequence around [Fe/H] = 0.2. 

The majority of high [$\alpha$/Fe] stars are bluer than the low [$\alpha$/Fe] subsample. Alternatively, at a given (NUV - G)$_0$, the high [$\alpha$/Fe] stars have a lower [Fe/H] than their low [$\alpha$/Fe] counterparts. While none of the low [$\alpha$/Fe] stars are metal poor enough to be considered halo stars (typically metallicities of [Fe/H] $<$ -0.5), they are not significantly different from the entire sample in other quantities except [$\alpha$/Fe]. The trends for the two different populations present in Figure 5 suggest different color-metallicity relations between them. 

\begin{figure*}[] % Fig6 - FeH vs NUVG
\centering
\includegraphics[width=1\textwidth]{f6.pdf}
\caption{[Fe/H] vs (NUV - G)$_0$ colored by [$\alpha$/Fe] without (left) and with (right) a Galactic extinction correction. We find several outliers at (NUV - G)$_0$ $<$ 6. \textbf{One suggestion is that these objects are binaries but that is uncertain. What we do know is that these objects populate a region in CMD space that is outside of the typical RC population seen in Figure 1.} The middle panels show low [$\alpha$/Fe] stars and the bottom panels show [$\alpha$/Fe] high stars. Each fit is done only on the data in that panel. We include all six fits in their corresponding panels. The (NUV - G)$_0$ spread over 4 magnitudes is much greater than in optical colors. The color-metallicity relation has a weak dependence on [$\alpha$/Fe]. The trend tightens when corrected for Galactic extinction.}
\end{figure*}

In Figure 6 we show a clear trend between (NUV - G)$_0$ color and metallicity for the full sample that spans a wider range of color than in optical wavelengths as shown in Figure 1. This relation becomes much tighter when an extinction correction is added. We also separate the two low and high [$\alpha$/Fe] populations in the middle and bottom panels, respectively. We obtain the following relationship for the full extinction-corrected sample:

\begin{equation}
    [Fe/H] = 0.256 (NUV - G)_0 - 2.204.
\end{equation}

The standard deviation from the linear fit is $\sigma_{\rm{[Fe/H]}}$ $\sim$ 0.16 dex. Equation 1 provides a new means to determine metallicity from photometry with a precision similar to low-resolution spectroscopy (e.g. SDSS SEGUE, \citealt{lee2011}, where they measure [Fe/H] to a precision of 0.23 dex) but at a much cheaper cost and can be obtained for many more stars. \textbf{The standard deviations of the fits to the low and high [$\alpha$/Fe] subsamples are 0.146 and 0.12 respectively. If we fit T$_{\rm{eff}}$ to model [Fe/H] instead of (NUV - G)$_0$, we get standard deviations for the full, low and high [$\alpha$/Fe] samples of 0.192, 0.154 and 0.111, the first two being larger than their (NUV - G)$_0$ fit counterparts and the high [$\alpha$/Fe] standard deviation being roughly the same.}

The two different low and high [$\alpha$/Fe] populations from Figure 5 appear to have different color-metallicity relations. High [$\alpha$/Fe] stars have much less scatter from the relation than low [$\alpha$/Fe] stars and the slope is higher in the high [$\alpha$/Fe] color-metallicity relation. At the very metal poor end ([Fe/H] $<$ -0.6) RC stars appear to be a part of the galaxy's thick disk (\citealt{brook12}, \citealt{hawkins15}) and from Figure 5 they are bluer objects in general. The low [$\alpha$/Fe] population also shows objects that are bluer than expected, including several outliers bluer than (NUV - G)$_0$ $<$ 6, some of which could be binaries. We will leave a more detailed discussion of these outliers to future papers. Even with the presence of outliers, the overall relation is still about as precise as spectroscopy. \textbf{If we apply the same analysis to just the \citealt{bovy14} sample (which the \citealt{ting18} sample does contain) then we get a standard deviation of $\sigma$ = 0.12. This is better than the full sample but using only 208 objects to fit versus our full sample. The scatter in the relation and overall shape of $\Delta$[Fe/H] is still very similar to that of the full sample.}

\begin{figure}[] % Fig7 - FeH vs NUVG test sample
\centering
\includegraphics[width=0.5\textwidth]{f7.pdf}
\caption{[Fe/H] vs (NUV - G)$_0$ for stars within the RC box described in the text. The blue line is the dust-corrected fit for the entire sample from figure 6, top right panel while the orange line is the fit to the test set. The data match the trend we see at (NUV - G)$_0$ $>$ 7.5. The slopes of the fits are the same with an offset in the y-intercept.}
\end{figure}

\begin{figure}[] % Fig8 - FeH vs NUVG using alpha/fe terms
\centering
\includegraphics[width=0.5\textwidth]{f8.pdf}
\caption{\textbf{[Fe/H] vs (NUV - G)$_0$ using a fit that includes a linear [$\alpha$/Fe] term.}}
\end{figure}


\subsection{Photometrically Selected Sample}
To demonstrate the effectiveness of this color-metallicty relation, we select a subset of photometrically-defined RC stars from the CMD in Figure 1. We define our RC box as bound at 6 $>$ (NUV - G)$_0$ $>$ 10.5 and 0.9 $>$ M$_G$ $>$ -0.1. This region contains many possible RC candidates from which one can also derive asteroseismic parameters with minimal contamination from RGB stars using the methods in \citealt{hawkins18}. The main source of bias comes from uncertainties in NUV which would restrict the range of objects detectable by GALEX. \textbf{If we assume the contamination in our sample is greater and only use this color cut to define our RC stars we can see how well this choice of box limits performs. We apply ASPCAPFLAG $=$ 0 and log g $<$ 3 to our catalog and get a subsample of 4,656 RC stars, or 90$\%$.}

We apply the same cuts as described in section 2.2 to a matched catalog between GAIS, Gaia DR2, and APOGEE DR14 and replot the [Fe/H] - (NUV - G)$_0$ relation with these data in Figure 7. The overall trend between ultraviolet-optical color and metallicity closely matches that of the spectroscopically-obtained RC sample. The scatter in the relation is larger but this is likely due to contamination from other giant stars. There is an offset at (NUV - G)$_0$ $<$ 8 for the highest [$\alpha$/Fe] stars which likely is due to the steeper relation for the high [$\alpha$/Fe] stars in this sample. \textbf{If we fit this test sample but separate them by their low and high [$\alpha$/Fe] components, we get standard deviations of 0.21, 0.21 and 0.16 for the whole, low and high [$\alpha$/Fe] samples respectively. As is seen in Figure 6, the high [$\alpha$/Fe] subsample tends to model the data better.}

Photometric-metallicity relations have been calculated or observed in the past. For example, \citealt{ivezic2008} use F and G main-sequence stars to derive a relation between [Fe/H], u - g and g - r. The u - g color, or the UV excess, depends on metallicity because of the high absorption of metals at bluer colors, affecting the star's flux. This UV excess depends on the g - r color which is related to the star's effective temperature. RC stars are known to have a flux-temperature relation that varies greatly depending on the metallicity. Metal line blanketing may also play a role due to the high [Fe/H] values in this sample (\citealt{girardi16}, \citealt{choi16}). Metals in stellar atmospheres absorb blue light due to metal line blanketing and should show significant absorption in bluer wavelengths like NUV. In our sample these metal-rich stars are the reddest, with (NUV - G)$_0$ values reaching up to 10 magnitudes versus the blue end at (NUV - G)$_0$ = 6 containing the most metal poor stars (Figure 6). Due to their similarity, these trends may also hold for stars along the giant branch albeit with increased scatter \textbf{within our empirical data due to the different physical properties between the whole range of giant branch stars versus the narrow red clump}. \textbf{We attempted to model this photometric-metallicity relation using [$\alpha$/Fe] terms in a variety of ways (similar to the method in \citealt{ivezic2008}), one of which is shown in Figure 8. These fits are worse than our relation that uses only a linear (NUV - G)$_0$ term.}

\subsection{Comparison to Models}
Finally we explore how our relation compares to predictions from a recently developed stellar evolutionary code. The MESA Isochrones and Stellar Tracks (MIST) model provides tracks and photometric outputs for a full range of stellar masses and metallicities with sufficient resolution to follow short-lived evolutionary stages (\citealt{choi16}). {\bf We used evolutionary tracks for the range of masses (1 to 2.5 Msolar) and metallicities ( -1.0 $<$ [Fe/H] $<$ 0.4; 0.25 dex steps) likely to appear in the Milky Way field RC population. Our models assume Solar abundances, and therefore do not include a range of alpha enhancement.} Luminosities were calculated in G, G$_{BP}$, G$_{RP}$, and NUV bands, using the most up to date Gaia bandpasses. We applied a photometric selection to the final outputs, selecting stars in a similar region for the RC in the CMD, as described above, while also restricting the model to the core helium-burning phase (MESA EEP 631-707). {\bf These cuts allow a wide range of stellar masses but we find that the metallicity vs. NUV-G color relation in the models is only weakly dependent on initial mass and age during this stage. Our selection cuts also include at least part of the region of the CMD known as the ``secondary red clump'' though we leave discussion of the distinct red clumps to future work.} In Figures 11 and 12 we plot NUV vs [Fe/H] colored by T$_{\rm{eff}}$ and log g, respectively. The extinction-corrected fits for the full sample and high [$\alpha$/Fe]-only subsample are shown on both plots.

The linear fits in Figure 10 provide a good match to the models over most of the metallicity range of our RC sample. Bluer stars tend to be hotter and have a lower metallicity and vice versa. The fit lines overlap models with the T$_{\rm{eff}}$ range in Figure 4. \textbf{Figure 9 shows that our mean fits are consistent with models with log g $\sim$ 2.4, with the high [$\alpha$/Fe] stars having a slightly lower log g (and T$_\rm{eff}$) at fixed metallicity, though the difference in log g between the mean of the two populations is very small (about 0.05)}. The larger scatter with increasing metallicity in the models is also seen in Figure 6, most notably in the low [$\alpha$/Fe] subsample. {\bf The scatter can be at least partially explained by scatter seen in the stellar evolution models for a range of stellar masses and ages. Other factors, such as errors in Galactic extinction, binarity, and a range of alpha enhancement may also contribute to the scatter. This topic will be studied in future papers.}

RC stars are used as standard candles in infrared due to their constant absolute magnitude and color. Metallicity, mass, age, and extinction make their use as standard candles difficult in bluer wavelengths. Using the color-metallicity relation we can create a metallicity map of the Galaxy (e.g. \citealt{onal16}) and increase the accuracy of RC stars as standard candles. These results also have implications on the use of RC stars as extinction probes. \citealt{yanchulova17} use HST observations that extend to the NUV and explain the spread in color as due to extinction. They conclude the RC is confined to a small region in the CMD with similar metallicities. \textbf{Instead, we see a large spread in (NUV - G)$_0$ with a range of metallicities,} indicating that metallicity may play a non-trivial role in understanding the RC in CMD space. \textbf{Their use as extinction probes and extinction mapping (seen in \citealt{girardi16} in areas such as the Large Magellanic Cloud and the Milky Way bulge) will vary depending on the star's metallicity and age, affecting the assumtion that there is no systematic variation in intrinsic RC properties.}

\begin{figure}[] % Fig9 - logg histograms
\centering
\includegraphics[width=0.5\textwidth]{f9.pdf}
\caption{\textbf{(a) log g histogram of our catalog showing a peak around 2.4. (b) log g vs log(T$_{eff}$). We see a clear trend relating log g, T$_{eff}$ with [Fe/H].}}
\end{figure}

\begin{figure}[] % Fig10 - FeH vs NUVG vs Teff
\centering
\includegraphics[width=0.4\textwidth]{f10.pdf}
\caption{Figure 6 plotted with MIST stellar evolution models colored by T$_{\rm{eff}}$. The blue line is the fit to the entire sample and the red line is the fit to only the high [$\alpha$/Fe] stars. {\bf Models are calculated using cuts described in text at discrete initial metallicities, spaced by 0.25 dex, though the bands in the plots are dithered for clarity. These plots are purely illustrative of the spread of color vs metallicity vs mass, etc. They do not correct or cut based on the lifetime of the star in each evolutionary phase and therefore overrepresent some parts of the population.}}
\end{figure}

\begin{figure}[] % Fig11 - FeH vs NUVG vs logg
\centering
\includegraphics[width=0.4\textwidth]{f11.pdf}
\caption{Figure 6 plotted with MIST stellar evolution models colored by log g. The blue and red lines {\bf and dithering of points for clarity} are the same as in Figure 10.}
\end{figure}


\begin{deluxetable*}{ccccccccccc}
%\centering
\tablehead{\colhead{RA} & \colhead{Dec} & \colhead{NUV} & \colhead{(NUV - G)$_0$} & \colhead{G$_{BP}$} & \colhead{(G$_{BP}$ - G$_{RP}$)$_0$} & \colhead{E(B - V)} & \colhead{DM} & \colhead{[Fe/H]} & \colhead{T$_{\rm{eff}}$} & \colhead{[$\alpha$/Fe]} \\ 
\colhead{(\deg)} & \colhead{(\deg)} & \colhead{(mag)} & \colhead{(mag)} & \colhead{(mag)} & \colhead{(mag)} & \colhead{(mag)} & \colhead{(mag)} & \colhead{} & \colhead{(K)} & \colhead{} } 
\startdata
0.1300 & 15.2717 & 18.11 $\pm$ 0.03 & 7.46 $\pm$ 0.03 & 11.17 $\pm$ 0.00 & 1.16 $\pm$ 0.00 & 0.04 & 10.41 & -0.41 $\pm$ 0.01 & 4931.00 & 0.10 $\pm$ 0.02 \\
0.2081 & 16.3654 & 19.40 $\pm$ 0.07 & 8.82 $\pm$ 0.07 & 11.18 $\pm$ 0.00 & 1.31 $\pm$ 0.00 & 0.04 & 10.13 & 0.06 $\pm$ 0.01 & 4699.00 & 0.01 $\pm$ 0.01 \\
0.3194 & 15.3927 & 19.49 $\pm$ 0.06 & 9.78 $\pm$ 0.06 & 10.36 $\pm$ 0.00 & 1.39 $\pm$ 0.00 & 0.04 & 9.10 & 0.34 $\pm$ 0.01 & 4553.00 & 0.02 $\pm$ 0.01 \\
0.4016 & 0.2359 & 18.97 $\pm$ 0.04 & 8.80 $\pm$ 0.04 & 10.74 $\pm$ 0.00 & 1.26 $\pm$ 0.00 & 0.02 & 9.61 & 0.04 $\pm$ 0.01 & 4723.00 & -0.01 $\pm$ -0.42 \\
0.5204 & 15.0377 & 19.08 $\pm$ 0.05 & 7.55 $\pm$ 0.05 & 12.04 $\pm$ 0.00 & 1.17 $\pm$ 0.00 & 0.05 & 11.27 & -0.36 $\pm$ 0.01 & 4915.00 & 0.13 $\pm$ 0.02 \\
0.6003 & 16.4273 & 17.30 $\pm$ 0.02 & 7.94 $\pm$ 0.02 & 9.90 $\pm$ 0.00 & 1.20 $\pm$ 0.00 & 0.03 & 9.03 & -0.24 $\pm$ 0.01 & 4826.00 & 0.03 $\pm$ 0.02 \\
0.6012 & 16.9140 & 19.10 $\pm$ 0.06 & 8.48 $\pm$ 0.06 & 11.18 $\pm$ 0.00 & 1.24 $\pm$ 0.00 & 0.03 & 10.17 & -0.01 $\pm$ 0.01 & 4782.00 & 0.03 $\pm$ 0.06 \\
0.6975 & 17.1347 & 17.58 $\pm$ 0.03 & 8.38 $\pm$ 0.03 & 9.77 $\pm$ 0.00 & 1.24 $\pm$ 0.00 & 0.03 & 9.25 & -0.23 $\pm$ 0.01 & 4734.00 & 0.02 $\pm$ 0.01 \\
0.9191 & 16.9788 & 18.68 $\pm$ 0.05 & 8.95 $\pm$ 0.05 & 10.37 $\pm$ 0.00 & 1.37 $\pm$ 0.00 & 0.03 & 9.18 & 0.17 $\pm$ 0.01 & 4548.00 & 0.05 $\pm$ 0.04 \\
0.9634 & 75.8578 & 19.44 $\pm$ 0.34 & 8.37 $\pm$ 0.34 & 11.79 $\pm$ 0.00 & 1.53 $\pm$ 0.00 & 0.30 & 10.67 & 0.15 $\pm$ 0.01 & 4929.00 & -0.01 $\pm$ -0.01 \\
\enddata
\tablecomments{(1) Gaia RA, (2) Gaia Dec, (3) GALEX NUV, (4) dust corrected NUV - G, (5) Gaia G$_{BP}$, (6) dust corrected G$_{BP}$ - G$_{RP}$, (7) E(B - V) from \citealt{GSF15}, (8) Distance Modulus, (9) stellar metallicity, (10) the effective temperature from APOGEE $\pm$ 91.47 K for all values, (11) alpha abundance. Table 1 is published in its entirety in the machine-readable format. A portion is shown here for guidance regarding its form and content.}
\end{deluxetable*}

\section{Conclusion}
Using a sample of 5,175 RC stars from APOGEE with data from GALEX and Gaia, we identify the RC in UV-optical CMD space as well as the existence of a color-metallicity relation that is tighter in (NUV - G)$_0$ than (G$_{BP}$ - G$_{RP}$)$_0$.  We see a strong dependence of color on T$_{\rm{eff}}$ and metallicity. As part of this analysis, we apply a Galactic extinction correction using a 3-D dust map from \citealt{GSF15} and Gaia distances, which further tightens the relation. If we separate the sample into low and high [$\alpha$/Fe] stars, high [$\alpha$/Fe] stars appear bluer than their low [$\alpha$/Fe] counterparts for a given temperature and redder at a fixed metallicity. Finally, we find a tight relation between (NUV - G)$_0$ and [Fe/H] with a standard deviation of about $\sigma$ = 0.16 that can be used to estimate stellar metallicities of RC stars when a spectroscopic metallicity measurement is missing. This relation will be used to obtain photometric metallicities from other stars in the same CMD space as RC candidates using only their UV-optical color. 
An NUV GALEX Plane Survey (Mohammed et al., in prep) will provide NUV measurements for the Galactic Plane for the first time using GALEX. This survey will provide millions of new objects brighter than NUV = 20 magnitude that will aid in RC investigations as well as many other fields in Galactic astronomy. Spectroscopic followup of these candidates could confirm their RC status using the method of \citealt{hawkins18} and \citealt{ting18} and allow us to further understand the UV-optical color-metallicity-age relation. RC stars are excellent extinction probes and if their metallicity is known it is enough to use a CMD to fit for its extinction values. Using our color-metallicity relation in conjunction with extinction measurements from \citealt{GSF15} we can narrow the variables to mass and age. 

\acknowledgments
We acknowledge support from NASA-ADP Grant NNX12AI50G. GALEX ($Galaxy$ $Evolution$ $Explorer$) is a NASA Small Explorer, launched in 2003 April. This work is based on data from the European Space Agency (ESA) mission Gaia (https://www.cosmos.esa.int/gaia), processed by the Gaia Data Processing and Analysis Consortium (DPAC, https://www.cosmos.esa.int/web/gaia/dpac/consortium). This study makes use of the publicly released data from APOGEE DR14. This research made use of Astropy, a community-developed core Python package for Astronomy (\citealt{astropy}). We acknowledge Dustin Lang for timely production of the Gaia DR2-GALEX catalog.

\bibliographystyle{aasjournal}
\bibliography{ms.bib}

\end{document}
